% YAAC Another Awesome CV LaTeX Template
%
% This template has been downloaded from:
% https://github.com/darwiin/yaac-another-awesome-cv
%
% Author:
% Christophe Roger
%
% Template license:
% CC BY-SA 4.0 (https://creativecommons.org/licenses/by-sa/4.0/)
%Section: Work Experience at the top
\sectionTitle{}{Work Experience}
%\faSuitcase
%\renewcommand{\labelitemi}{$\bullet$}
\begin{experiences}
%%%----------------------------------------------------------------------------------------EXPERIENCE 5
 \experience
    {The College Board $\cdot$ Newtown, PA}{Present }
    {Psychometrician}
    {July 2019 }    {
       \begin{itemize}
	\item Wrote R script to simulate responses to dependent items for a larger study that was conducted to decide on the type of response model that should be used.    
	\item Conducted a simulation study using R on an item statistical fit criteria called $s-x^2$ to find parameters that could be used to quickly filter through poorly fitting items, reducing item review time by as much as 90\%.  
	\item Conducted a simulation study using R to investigate differences on various scoring rules for missing item responses, which served as evidence for using a specific scoring rule over the others.
	\item Wrote R code to quickly calculate, organize and present statistics from various sources to help leadership make final decisions on raw score to scale score conversions for all 32 AP tests in an extremely short timeline.
	\item Organized and led training sessions to help psychometricians and statisticians conduct special psychometric tasks.   
	\item Wrote standard operating procedure and job aids to help guide staff conduct item response theory item calibration on trial test items. 
	\item Provided support as needed on various psychometric tasks such as test item analysis, differential item analysis, test security analysis, and test equating.
  
       \end{itemize}
                    }

  \emptySeparator  


%----------------------------------------------------------------------------------------EXPERIENCE 4
	\experience
    {Center for Applied Linguistics $\cdot$ Washington, DC}{July 2019}
    {Senior Research Associate/Psychometrician}
    {June 2017} {
      \begin{itemize}

            \item Led the quantitative research team to execute analyses like equating, item calibration, and standard setting for successful submission to the Office of Career and Technical Education for approval of a computer-adaptive speaking, reading, and writing test taken by thousands of second language learners in the United States.
      
            \item Developed an R script to automatically configure and execute a test classification accuracy software for 192 different samples of approximately 8,000 students, reducing overall analysis time by 80\% . 

          \item Collaborated with the test development team and conducted all quantitative analyses for publication of CAL-EPT test in Mexico, giving approximately 8,000 secondary students access to a reliable and valid English language test since FY 2018.
          
          \item Simulated examinee polytomously scored responses to create distributions of possible scores, which served as the main methodology for deciding on the optimal routing criteria for a paper-based version of a multi-stage computer-adaptive test. 
          
          \item Built a data reshaping package and created a user-friendly procedure to help non-quantitative team members clean messy data in approximately 15 minutes, reducing process time by 75\%.
          
          \item Developed an R script to automatically configure and execute item calibration software for 80 different subset of items, reducing analysis time by 80\%.
          
      \end{itemize}
      

                    }
    \emptySeparator                
                    
    \experienceADD
    {May 2017}
    {Psychometrics and Quantitative Research Intern}
    {August 2015} {
      \begin{itemize}
            \item Created an R package to conduct item bias analysis (DIF) for use specifically with multi-stage adaptive tests, allowing for the continued delivery of crucial item statistics to one of our major clients.
            
           \item Developed an R function to identify the combinations of routes taken by examinees taking a multi-stage adaptive test for reading and listening, providing important ancillary information to help decide how item reliability measures for multi-stage adaptive tests should be calculated.

      \end{itemize}
                    }

   \emptySeparator
   
%%%----------------------------------------------------------------------------------------EXPERIENCE 3   
 \experience
    {University of Maryland $\cdot$ College Park, MD}{May 2015}
    {Research Assistant}
    {August 2013}    {
       \begin{itemize}
             \item Compared methods for handling attrition in longitudinal datasets, leading to a journal publication that is currently in press for the Journal of Experimental Education.   
             
             \item Conducted statistical analysis of linear and non-linear longitudinal models that resulted in a book chapter on handling attrition in longitudinal survey data.       
       \end{itemize}
                    }
%  %{R, SAS, Excel, \LaTeX, Longitudinal Modeling}
  \emptySeparator  
  
  \experienceADD{August 2015}
  {Lecturer}
  {January 2015}{
       \begin{itemize}
             \item Taught introductory statistics to approximately 40 students in the spring and summer.
             \item Wrote tutorials for students to easily learn how to conduct statistical analyses using SPSS.
      \end{itemize}
  }
\emptySeparator 
%%%----------------------------------------------------------------------------------------EXPERIENCE 2
 \experience
    {Maryland Assessment Research Center $\cdot$ College Park, MD}{May 2015}
    {Research Assistant}
    {August 2013}    {
       \begin{itemize}
             \item Executed a student diagnostic model comparison study that led to an online publication and timely submission to the Maryland State Department of Education %(\url{https://marces.org/current/ExecutiveReport_MARC_2014_Cognitive\%20Diagnostic\%20Models.pdf}).           
             \item Drafted a literature review for the Maryland Department of Education on statistics-related issues involving the Maryland Student Achievement test (\url{https://marces.org/current/ExecutiveReport\_MARC\_2014\_Cognitive DiagnosticModels.pdf}).
       \end{itemize}
                    }

  \emptySeparator      

%   %----------------------------------------------------------------------------------------EXPERIENCE 1  
  \experience
  {Bronxwood Preparatory High School $\cdot$ Bronx, NY}{June 2010}
  {Teacher}
 {September 2007}   {
      \begin{itemize}
            \item Helped low-income English language learners, non-English language learners, and special education students pass the NY State Algebra Regents exam, maintaining a 60\% pass rate for 3 years. 
            \item Tutored 10 English language learners to help them pass the NY State Algebra Regents exam in Spanish.
      \end{itemize}
                  }
\end{experiences}
